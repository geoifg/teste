% Options for packages loaded elsewhere
\PassOptionsToPackage{unicode}{hyperref}
\PassOptionsToPackage{hyphens}{url}
%
\documentclass[
]{book}
\usepackage{lmodern}
\usepackage{amssymb,amsmath}
\usepackage{ifxetex,ifluatex}
\ifnum 0\ifxetex 1\fi\ifluatex 1\fi=0 % if pdftex
  \usepackage[T1]{fontenc}
  \usepackage[utf8]{inputenc}
  \usepackage{textcomp} % provide euro and other symbols
\else % if luatex or xetex
  \usepackage{unicode-math}
  \defaultfontfeatures{Scale=MatchLowercase}
  \defaultfontfeatures[\rmfamily]{Ligatures=TeX,Scale=1}
\fi
% Use upquote if available, for straight quotes in verbatim environments
\IfFileExists{upquote.sty}{\usepackage{upquote}}{}
\IfFileExists{microtype.sty}{% use microtype if available
  \usepackage[]{microtype}
  \UseMicrotypeSet[protrusion]{basicmath} % disable protrusion for tt fonts
}{}
\makeatletter
\@ifundefined{KOMAClassName}{% if non-KOMA class
  \IfFileExists{parskip.sty}{%
    \usepackage{parskip}
  }{% else
    \setlength{\parindent}{0pt}
    \setlength{\parskip}{6pt plus 2pt minus 1pt}}
}{% if KOMA class
  \KOMAoptions{parskip=half}}
\makeatother
\usepackage{xcolor}
\IfFileExists{xurl.sty}{\usepackage{xurl}}{} % add URL line breaks if available
\IfFileExists{bookmark.sty}{\usepackage{bookmark}}{\usepackage{hyperref}}
\hypersetup{
  pdftitle={Sistemas de Informaçãoes Geográficas para Transporte},
  pdfauthor={Édipo H. Cremon},
  hidelinks,
  pdfcreator={LaTeX via pandoc}}
\urlstyle{same} % disable monospaced font for URLs
\usepackage{color}
\usepackage{fancyvrb}
\newcommand{\VerbBar}{|}
\newcommand{\VERB}{\Verb[commandchars=\\\{\}]}
\DefineVerbatimEnvironment{Highlighting}{Verbatim}{commandchars=\\\{\}}
% Add ',fontsize=\small' for more characters per line
\usepackage{framed}
\definecolor{shadecolor}{RGB}{248,248,248}
\newenvironment{Shaded}{\begin{snugshade}}{\end{snugshade}}
\newcommand{\AlertTok}[1]{\textcolor[rgb]{0.94,0.16,0.16}{#1}}
\newcommand{\AnnotationTok}[1]{\textcolor[rgb]{0.56,0.35,0.01}{\textbf{\textit{#1}}}}
\newcommand{\AttributeTok}[1]{\textcolor[rgb]{0.77,0.63,0.00}{#1}}
\newcommand{\BaseNTok}[1]{\textcolor[rgb]{0.00,0.00,0.81}{#1}}
\newcommand{\BuiltInTok}[1]{#1}
\newcommand{\CharTok}[1]{\textcolor[rgb]{0.31,0.60,0.02}{#1}}
\newcommand{\CommentTok}[1]{\textcolor[rgb]{0.56,0.35,0.01}{\textit{#1}}}
\newcommand{\CommentVarTok}[1]{\textcolor[rgb]{0.56,0.35,0.01}{\textbf{\textit{#1}}}}
\newcommand{\ConstantTok}[1]{\textcolor[rgb]{0.00,0.00,0.00}{#1}}
\newcommand{\ControlFlowTok}[1]{\textcolor[rgb]{0.13,0.29,0.53}{\textbf{#1}}}
\newcommand{\DataTypeTok}[1]{\textcolor[rgb]{0.13,0.29,0.53}{#1}}
\newcommand{\DecValTok}[1]{\textcolor[rgb]{0.00,0.00,0.81}{#1}}
\newcommand{\DocumentationTok}[1]{\textcolor[rgb]{0.56,0.35,0.01}{\textbf{\textit{#1}}}}
\newcommand{\ErrorTok}[1]{\textcolor[rgb]{0.64,0.00,0.00}{\textbf{#1}}}
\newcommand{\ExtensionTok}[1]{#1}
\newcommand{\FloatTok}[1]{\textcolor[rgb]{0.00,0.00,0.81}{#1}}
\newcommand{\FunctionTok}[1]{\textcolor[rgb]{0.00,0.00,0.00}{#1}}
\newcommand{\ImportTok}[1]{#1}
\newcommand{\InformationTok}[1]{\textcolor[rgb]{0.56,0.35,0.01}{\textbf{\textit{#1}}}}
\newcommand{\KeywordTok}[1]{\textcolor[rgb]{0.13,0.29,0.53}{\textbf{#1}}}
\newcommand{\NormalTok}[1]{#1}
\newcommand{\OperatorTok}[1]{\textcolor[rgb]{0.81,0.36,0.00}{\textbf{#1}}}
\newcommand{\OtherTok}[1]{\textcolor[rgb]{0.56,0.35,0.01}{#1}}
\newcommand{\PreprocessorTok}[1]{\textcolor[rgb]{0.56,0.35,0.01}{\textit{#1}}}
\newcommand{\RegionMarkerTok}[1]{#1}
\newcommand{\SpecialCharTok}[1]{\textcolor[rgb]{0.00,0.00,0.00}{#1}}
\newcommand{\SpecialStringTok}[1]{\textcolor[rgb]{0.31,0.60,0.02}{#1}}
\newcommand{\StringTok}[1]{\textcolor[rgb]{0.31,0.60,0.02}{#1}}
\newcommand{\VariableTok}[1]{\textcolor[rgb]{0.00,0.00,0.00}{#1}}
\newcommand{\VerbatimStringTok}[1]{\textcolor[rgb]{0.31,0.60,0.02}{#1}}
\newcommand{\WarningTok}[1]{\textcolor[rgb]{0.56,0.35,0.01}{\textbf{\textit{#1}}}}
\usepackage{longtable,booktabs}
% Correct order of tables after \paragraph or \subparagraph
\usepackage{etoolbox}
\makeatletter
\patchcmd\longtable{\par}{\if@noskipsec\mbox{}\fi\par}{}{}
\makeatother
% Allow footnotes in longtable head/foot
\IfFileExists{footnotehyper.sty}{\usepackage{footnotehyper}}{\usepackage{footnote}}
\makesavenoteenv{longtable}
\usepackage{graphicx,grffile}
\makeatletter
\def\maxwidth{\ifdim\Gin@nat@width>\linewidth\linewidth\else\Gin@nat@width\fi}
\def\maxheight{\ifdim\Gin@nat@height>\textheight\textheight\else\Gin@nat@height\fi}
\makeatother
% Scale images if necessary, so that they will not overflow the page
% margins by default, and it is still possible to overwrite the defaults
% using explicit options in \includegraphics[width, height, ...]{}
\setkeys{Gin}{width=\maxwidth,height=\maxheight,keepaspectratio}
% Set default figure placement to htbp
\makeatletter
\def\fps@figure{htbp}
\makeatother
\setlength{\emergencystretch}{3em} % prevent overfull lines
\providecommand{\tightlist}{%
  \setlength{\itemsep}{0pt}\setlength{\parskip}{0pt}}
\setcounter{secnumdepth}{5}
\usepackage{booktabs}
\usepackage[]{natbib}
\bibliographystyle{apalike}

\title{Sistemas de Informaçãoes Geográficas para Transporte}
\author{Édipo H. Cremon}
\date{2020-11-16}

\begin{document}
\maketitle

{
\setcounter{tocdepth}{1}
\tableofcontents
}
\hypertarget{introduuxe7uxe3o-ao-sig}{%
\chapter{Introdução ao SIG}\label{introduuxe7uxe3o-ao-sig}}

É comum nos depararmos com uma série de nomenclaturas cuja diferenciação torna-se nebulosa até mesmo para quem está atuando na área há bastante tempo. É comum vermos pessoas usando os termos geoprocessamento, geotecnologias e SIG (Sistema de Informação Geográfica) como sinônimos. Mas não são! Inclusive há certa hierarquia dentro destes conceitos.

\textbf{Geoprocessamento}

Corresponde ``a área do conhecimento que utiliza técnicas matemáticas e computacionais para o tratamento da informação geográfica e que vem influenciando de maneira crescente as áreas de Cartografia, Análise de Recursos Naturais, Transportes, Comunicações, Energia e Planejamento Urbano e Regional'' (Câmara e Davis, 2004).

\textbf{Geotecnologias}

\begin{itemize}
\item
  Conjunto de tecnologias para coleta, processamento, análise e disponibilização de informação com referência geográfica
\item
  As geotecnologias são compostas por soluções em hardware, software e peopleware (recursos humanos) que juntos se constituem em poderosas ferramentas para tomada de decisão
\item
  As geotecnologias estão entre os três mercados emergentes mais importantes da atualidade, junto com a nanotecnologia e a biotecnologia (Revista Nature, jan2004)
\end{itemize}

\textbf{Sistemas de Informação Geográfica - SIG}

As ferramentas computacionais para Geoprocessamento, chamadas de SIG, permitem realizar análises complexas, ao integrar dados de diversas fontes e ao criar bancos de dados georreferenciados. Tornam ainda possível automatizar a produção de documentos cartográficos (Câmara e Davis, 2004).

Podemos dizer que é convergência de diferentes disciplinas onde o espaço (computacionalmente representado) é a linguagem comum. Como inúmeros campos da ciência (geografia, geologia, agronomia, engenharias ambiental, florestal, cartográfica e agrimensura, do transporte, etc) tratam de dados com uma localização geográfica, logo o SIG se faz como uma poderosa ferramenta de análise destes campos.

Pode-se definir SIG como um conjunto de ferramentas computacionais composto de equipamentos e programas que por meio de técnicas, integra dados (das mais diversas fontes), pessoas e instituições, de forma a tornar possível a coleta, o armazenamento, a análise e a disponibilização, a partir de dados georreferenciados, de informações produzidas por meio de aplicações, visando maior facilidade, segurança e agilidade nas atividades humanas referentes ao monitoramento, planejamento e tomada de decisão relativas ao espaço geográfico.

\textbf{Informação Geográfica (IG)}

Por sua vez, a informação geográfica pode ser entendida como:

\begin{itemize}
\tightlist
\item
  Informação sobre lugares na superfície da Terra
\item
  Conhecimento sobre onde alguma coisa está
\item
  Conhecimento sobre o que está em uma dada localização (GOODCHILD, 1997)
\end{itemize}

Nesse contexto, o adjetivo ``geográfico'' corresponde a informação que está sobre a superfície da Terra. Dado essa restrição é comum ser usado o termo informação espacial, uma vez que o termo espacial não se restringe apenas a informações sobre a superfície terrestre, se referindo genericamente a qualquer lugar no espaço. Em ambiente SIG é comum que diversas análises sejam denominadas de análise espacial, não análise geográfica, já que podem ser aplicadas a diversas finalidades que podem ser mais abrangentes que a algo que está sobre a superfície terrestre. Também é comum encontrar o termo informação geoespacial ou geoinformação que na prática tem a mesma conotação que informação espacial, ou seja, mais abrangente que informação geográfica.

\textbf{Ciência da Informação Geográfica}

Dado que a ciência da informação estuda os temas fundamentais decorrentes da criação, manuseio, armazenamento e uso da informação. Logo, a Ciência da Informação Geográfica estuda os temas decorrentes da IG.
Na bibliografia é comum constatar o uso dos termos ciência da informação geográfica, geomática, ciência da informação espacial e engenharia da geoinformação. De autor pra autor esses termos podem ter ligeiras diferenças, mas em linhas gerais são denominações muito próximas.

A ciência da informação geográfica está mais que consolidada no meio acadêmico e prova disso são eventos e revistas científicas dedicadas à sua temática. Como exemplo temos as revistas Geographical Information Science, Journal of Geographic Information System e o International Journal of Geo-Information.

\textbf{SIG-T}

O SIG permite a integração dados geográficos (georreferenciados) e não-geográficos, facilitando a coleta, o armazenamento, a análise e a disponibilização, a partir de dados georreferenciados. Na área de transporte, o SIG constitui em uma importante ferramenta no tratamento de toda a informação base (estatística ou cartográfica) referente às locações, às deslocações, aos fluxos e aos motivos pelos quais existe o transporte de algo. Dado esse potencial, há um aumento de aplicações de SIG no mercado e na academia, fazendo surgir o nome de SIG- T (em inglês: GIS-T - Geographical Information Systems in Transportation).

A consolidação do SIG-T como uma área científica pode ser observada na inserção do tema em revistas especializadas de transporte tradicionais (por exemplo, o Journal of Advanced Transportation e o Journal of Transportation Planning and Technology). Outro exemplo desta consolidação é a realização de eventos científicos voltados ao tema, com destaque é interessante mencionar o Annual GIS for Transportation Symposium.

This is a \emph{sample} book written in \textbf{Markdown}. You can use anything that Pandoc's Markdown supports, e.g., a math equation \(a^2 + b^2 = c^2\).

The \textbf{bookdown} package can be installed from CRAN or Github:

\begin{Shaded}
\begin{Highlighting}[]
\KeywordTok{install.packages}\NormalTok{(}\StringTok{"bookdown"}\NormalTok{)}
\CommentTok{# or the development version}
\CommentTok{# devtools::install_github("rstudio/bookdown")}
\end{Highlighting}
\end{Shaded}

Remember each Rmd file contains one and only one chapter, and a chapter is defined by the first-level heading \texttt{\#}.

To compile this example to PDF, you need XeLaTeX. You are recommended to install TinyTeX (which includes XeLaTeX): \url{https://yihui.org/tinytex/}.

\hypertarget{intro}{%
\chapter{Visão geral de um SIG}\label{intro}}

Há pelo menos três grandes maneiras de utilizar um SIG (Câmara e Queiroz, 2004 ):

\begin{itemize}
\tightlist
\item
  como ferramenta para produção de mapas
\item
  como suporte para análise espacial de fenômenos
\item
  como um banco de dados geográficos, com funções de armazenamento e recuperação de informação espacial.
\end{itemize}

A facilidade de trabalhar com a informação geográfica em ambiente SIG, seja na sua criação e edição, tornou o trabalho dos cartógrafos mais facilitada, constituindo uma importante ferramenta tanta na cartografia sistemática, quanto na cartografia temática. Com as ferramentas robustas de visualização, simbologia e layout, mapas tem sido produzidos tanto para saídas gráficas em meio digital e para impressão, com destaque para a criação de webmaps, onde os mapas são acessados via web interativamente com o usuário.

O que torna o SIG um ambiente poderoso de trabalho em relação a pacotes dedicados a desenho (p.ex. CAD) é sua capacidade de análise espacial. Baseado em inúmeras ferramentas é possível cruzar, interpolar e agregar dados para se chegar novas informações, tendo a inferência geográfica como um grande campo a ser explorado. A disciplina de SIG-2 do nosso curso abordará em mais detalhe esse conteúdo.
Por fim, e não menos importante, deve-se destacar a capacidade de trabalhar com SIG como um ambiente gerenciamento de banco de dados geográficos. Em tempos, onde há um volume considerável de informações, trabalhar com banco de dados é imprescindível, se for dados geográficos, um banco de dados geográficos é mais relevante ainda. Os softwares de SIG permitem gerenciar esses bancos de dados com funções de armazenamento e recuperação da informação que facilitam e muito quando trabalhamos com grande quantidade de dados.

Por fim, e não menos importante, deve-se destacar a capacidade de trabalhar com SIG como um ambiente gerenciamento de banco de dados geográficos. Em tempos, onde há um volume considerável de informações, trabalhar com banco de dados é imprescindível, se for dados geográficos, um banco de dados geográficos é mais relevante ainda. Os softwares de SIG permitem gerenciar esses bancos de dados com funções de armazenamento e recuperação da informação que facilitam e muito quando trabalhamos com grande quantidade de dados.

Nesse sentido, é possível indicar que as principais características de SIG são:

\begin{itemize}
\tightlist
\item
  Inserir e integrar, numa única base de dados, informações espaciais provenientes de dados cartográficos, dados censitários e cadastro urbano e rural, imagens de satélite, redes e modelos numéricos de terreno (informações geográficas).
\end{itemize}

\textbf{Componentes de um SIG}

Os clássicos componentes SIG são formados por cinco componentes, todos inter-relacionados entre si, sendo eles: Recursos humanos, dados, software, hardware e análises.

\textbf{Hardware}

Tradicionalmente, o hardware se limitava ao microcomputador onde o usuário interagia com as operações SIG, nisso incluía todas suas parte como o CPU, teclado, monitor, mouse e até acessórios extras como mesa digitalizadora, scanner, impressora, etc, mas atualmente isso é muito mais amplo. Laptops, tablets e smarphones, por exemplo vem sendo utilizados dentro do contexto de SIG.

\textbf{Software}

Outra peça dos componentes SIG é o software. Muitas pessoas associam o SIG diretamente ao software, mas o mais correto é dizer software (ou aplicativo) de SIG. Consistem nos programas computacionais em que de fato serão utilizados para manipular os dados no intuito de produzir informações geográficas. No geral esses programas são robustos e possuem inúmeras ferramentas para exibição dos dados e informações geográficas, ferramentas para realizar edição, reprojeção e cruzamento dos dados geográficos, entre outras.

De acordo com Longley et al (2013), os softwares de SIG podem ser divididos nos seguintes tipos:

\begin{itemize}
\tightlist
\item
  Desktop
\item
  Mapeamento na Web
\item
  Servidor
\item
  Globo Virtual
\item
  Para desenvolvedores
\item
  Portátil
\item
  Outros
\end{itemize}

\textbf{Dados}
Em SIG é possível trabalhar tanto com dados geográficos, quanto com dados alfanuméricos. De modo genérico, os dados podem ser classificados em: espaciais e não espaciais

(\ldots)

\textbf{Análise}

Dentre as diversas funções analíticas que o SIG pode executar, podemos agrupar em:
• Gerenciamento do Banco de dados geográficos
• Análise espacial;
• Topologia;
• Processamento de imagens;
• Análise digital do terreno;
• Visualização/Plotagem.

\textbf{Recursos humanos}

Para fazer o SIG funcionar é essencial que pessoas operem. E qual seria o perfil dos profissionais de geotecnologias?
* Formação contínua (cursos, eventos, leitura, etc);
\emph{Rede de contatos e bom relacionamento pessoal (inclui colegas da universidade, trabalho e professores);
} Colaboração;
* Conhecimento em mais de um software;
* Conhecimento em alguma linguagem de programação é um diferencial (ex.: Python, R).

\begin{verbatim}
This text will appear styled different (for example)
\end{verbatim}

\begin{quote}
**SIG vs CAD
\end{quote}

\begin{quote}
Oferecer mecanismos para combinar as várias informações, através de algoritmos de manipulação e análise, bem como para consultar, recuperar, visualizar e plotar o conteúdo da base de dados georreferenciados (Câmara e Queiroz, 2004).
\end{quote}

You can label chapter and section titles using \texttt{\{\#label\}} after them, e.g., we can reference Chapter \ref{intro}. If you do not manually label them, there will be automatic labels anyway, e.g., Chapter \ref{methods}.

Figures and tables with captions will be placed in \texttt{figure} and \texttt{table} environments, respectively.

\begin{Shaded}
\begin{Highlighting}[]
\KeywordTok{par}\NormalTok{(}\DataTypeTok{mar =} \KeywordTok{c}\NormalTok{(}\DecValTok{4}\NormalTok{, }\DecValTok{4}\NormalTok{, }\FloatTok{.1}\NormalTok{, }\FloatTok{.1}\NormalTok{))}
\KeywordTok{plot}\NormalTok{(pressure, }\DataTypeTok{type =} \StringTok{'b'}\NormalTok{, }\DataTypeTok{pch =} \DecValTok{19}\NormalTok{)}
\end{Highlighting}
\end{Shaded}

\begin{figure}

{\centering \includegraphics[width=0.8\linewidth]{SIG-T_files/figure-latex/nice-fig-1} 

}

\caption{Here is a nice figure!}\label{fig:nice-fig}
\end{figure}

Reference a figure by its code chunk label with the \texttt{fig:} prefix, e.g., see Figure \ref{fig:nice-fig}. Similarly, you can reference tables generated from \texttt{knitr::kable()}, e.g., see Table \ref{tab:nice-tab}.

\begin{Shaded}
\begin{Highlighting}[]
\NormalTok{knitr}\OperatorTok{::}\KeywordTok{kable}\NormalTok{(}
  \KeywordTok{head}\NormalTok{(iris, }\DecValTok{20}\NormalTok{), }\DataTypeTok{caption =} \StringTok{'Here is a nice table!'}\NormalTok{,}
  \DataTypeTok{booktabs =} \OtherTok{TRUE}
\NormalTok{)}
\end{Highlighting}
\end{Shaded}

\begin{table}

\caption{\label{tab:nice-tab}Here is a nice table!}
\centering
\begin{tabular}[t]{rrrrl}
\toprule
Sepal.Length & Sepal.Width & Petal.Length & Petal.Width & Species\\
\midrule
5.1 & 3.5 & 1.4 & 0.2 & setosa\\
4.9 & 3.0 & 1.4 & 0.2 & setosa\\
4.7 & 3.2 & 1.3 & 0.2 & setosa\\
4.6 & 3.1 & 1.5 & 0.2 & setosa\\
5.0 & 3.6 & 1.4 & 0.2 & setosa\\
\addlinespace
5.4 & 3.9 & 1.7 & 0.4 & setosa\\
4.6 & 3.4 & 1.4 & 0.3 & setosa\\
5.0 & 3.4 & 1.5 & 0.2 & setosa\\
4.4 & 2.9 & 1.4 & 0.2 & setosa\\
4.9 & 3.1 & 1.5 & 0.1 & setosa\\
\addlinespace
5.4 & 3.7 & 1.5 & 0.2 & setosa\\
4.8 & 3.4 & 1.6 & 0.2 & setosa\\
4.8 & 3.0 & 1.4 & 0.1 & setosa\\
4.3 & 3.0 & 1.1 & 0.1 & setosa\\
5.8 & 4.0 & 1.2 & 0.2 & setosa\\
\addlinespace
5.7 & 4.4 & 1.5 & 0.4 & setosa\\
5.4 & 3.9 & 1.3 & 0.4 & setosa\\
5.1 & 3.5 & 1.4 & 0.3 & setosa\\
5.7 & 3.8 & 1.7 & 0.3 & setosa\\
5.1 & 3.8 & 1.5 & 0.3 & setosa\\
\bottomrule
\end{tabular}
\end{table}

You can write citations, too. For example, we are using the \textbf{bookdown} package \citep{R-bookdown} in this sample book, which was built on top of R Markdown and \textbf{knitr} \citep{xie2015}.

\hypertarget{literature}{%
\chapter{Literature}\label{literature}}

Here is a review of existing methods.

\hypertarget{methods}{%
\chapter{Methods}\label{methods}}

We describe our methods in this chapter.

\hypertarget{applications}{%
\chapter{Applications}\label{applications}}

Some \emph{significant} applications are demonstrated in this chapter.

\hypertarget{example-one}{%
\section{Example one}\label{example-one}}

\hypertarget{example-two}{%
\section{Example two}\label{example-two}}

\hypertarget{final-words}{%
\chapter{Final Words}\label{final-words}}

We have finished a nice book.

  \bibliography{book.bib,packages.bib}

\end{document}
